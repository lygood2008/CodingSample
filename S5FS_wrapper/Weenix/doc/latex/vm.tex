\chapter{Virtual Memory}
\label{vm}

\section{Introduction}

At this point, your Weenix contains a threading library, some thin wrappers around device drivers, and basic file system support with a caching layer. By the end of this assignment, Weenix will be a full operating system. With the addition of virtual memory, your kernel will start managing user address spaces, running user-level code, and servicing system calls. After completing this project, everything you did before will seem insignificant.

This assignment is substantial, and also very prone to difficult bugs. Before you begin, make sure the rest of your kernel is functioning exactly as you expect. You will undoubtedly uncover bugs in old code throughout the course of this assignment, but minimizing the number you find before you start will be helpful. Make sure to start early and ask questions frequently; it is very easy to get lost in this assignment.

% TODO improve next paragraph

Because VM bugs can spring up in code you wrote months ago, this is where you will probably find out whether or not your implementations of the previous assignments are up to par. We would like to point out that there are several Weenix- and OS-specific debugging tools and techniques in Appendix \ref{Debugging} which will be \textit{extremely} useful if you have not been using them so far.

\section{Virtual Memory Maps}

The first thing you should do in this assignment is write the code for managing a process' virtual address space. The virtual address space for a process (also known as its ``memory map'') is stored as a linked list of virtual memory areas (also referred to as ``memory regions''), each of which correspond to some memory object which provides pages of memory to the process on demand. As you have likely already realized, this means that everything from files to disks can be mapped into the address space of Weenix processes, and it should now make even more sense why we used memory objects extensively in the last assignment instead of reading and writing directly to disk. Of course, some memory areas will not correspond to existing data (the stack and heap, for instance). We will explain how that works in the section on \wlink{anon}{anonymous objects}.

In order to manage address spaces, you must maintain each process' list of virtual memory areas. Each memory region is essentially a range of virtual page numbers, a memory object, and the page offset into that memory object that the first page of the virtual memory area corresponds to. Make sure that you understand why these numbers are all stored at page resolution instead of byte (address) resolution. You must keep the areas sorted by the start of their virtual page ranges and ensure that no two ranges overlap with each other. There will be several edge cases (which are better documented in the code) where you will have to unmap a section of a virtual memory area, which could require splitting an area or truncating the beginning or end of an area.

While there is very little conceptually difficult code to write in this section of the assignment, off-by-one bugs are extremely common and become very difficult to track down later on, so unit-testing this code is a good idea.

\section{Page Fault Handler}

After your memory maps are working, you will need a way to actually load the data into memory when a process attempts to access it. This is done by the page fault handler. The page fault handler is triggered by a processor interrupt when a process attempts to access an address for which it has no lookup entry in the page table or the permissions on that entry do not allow the type of access that is being attempted.

At this point in the project, any page faults that have occurred have resulted in a kernel panic. That is because Weenix does not support kernel-level page faults, meaning that the entire kernel address space must reside in memory at all times. This functionality is written into a wrapper for the page fault handler you will write which short-circuits kernel page faults. More details on how this function works can most easily be found in the code.

The combination of the page fault handler and the virtual memory maps should be enough to get a very simple page fault to occur in a userland program. At this point, you can set up a userland program to run from inside the \texttt{init} process by running \texttt{kernel\_execve()}, passing the path to any program on your (virtual) disk as an argument. Similar to the \texttt{exec()} system call, this will replace the memory map of the current process to set up another program to run, but it will be better than \texttt{exec()} in this case because the setup is done exclusively in kernel space, so it can be used before you have a fully functional userspace. When the program begins, it should cause a page fault to be generated. This is your first step towards having a functional userland.

A simple implementation of the page fault handler will be enough to start with, but eventually this will be a relatively logic-heavy function. First, the page fault handler should search for the virtual memory area containing the address that was faulted on. Then, the permissions of this area should be checked against the flags variable that is passed to the handler, which tells the handler whether the attempted access is a read or a write. If the memory area containing the accessed page is not found or the permissions would make this access illegal, the current process is killed with an exit status of \texttt{EFAULT}. Of course, if Weenix supported UNIX signals, it would send a \texttt{SIGSEGV} signal instead.

Once the virtual memory area is found, Weenix must search for the missing page and map it into the page table of the current process so that the access can be retried using the virtual address of the page that's being added and the physical address where it resides. Fetching the missing page will require a lot of help from the page frame caching system, namely for looking up the page and dirtying it if the access is a write. This, in turn, will rely on two new types of memory objects you will need to implement.

\subsection{The Memory Management Unit}

In order to map the virtual address to its corresponding page of memory, you will need to use the page table functions. A good portion of memory management is done for you, but you will have to fill in page table entries when page faults occur, flush the translation lookaside buffer (TLB) when necessary, and manage copy-on-write pages yourself. You will also need to make sure that pages which are not backed by files remain pinned, so they do not get paged out by accident.

\section{Memory Management}

As you have implemented it currently, the caching layer of Weenix works exclusively for pages of files or disks that have been mapped into memory. You will extend it by creating memory objects which will provide two new types of memory which are not backed by any on-disk structures.

\subsection{Anonymous Objects} \label{anon}

So far, you have used the memory objects of your block device and files to fill page frames as you needed data from disk, but it does not make sense to back some virtual memory areas, such as a process' stack, with data on disk. What you often want is objects which initialize pages by filling them with zeroes and pin their pages in memory as long as the process is using them. These are known as anonymous objects since they are not backed by any persistent data (which would have a filename associated with it). Anonymous objects are relatively simple to implement, so look for a better description of how they work in the code comments.

Notably, anonymous objects cannot be paged out in Weenix. The designers chose not to implement this feature because memory pressure will rarely be an issue in your operating system, and implementing a swap space is not terribly interesting or vital to implement as a result.

\subsection{Shadow Objects}

Anonymous objects are easy to implement, however you will also need a much more sophisticated form of memory object called a shadow object to implement \texttt{fork()}. These will be used to implement copy-on-write for privately-mapped blocks that are accessed after forking. Because of how involved shadow objects must be, you should refer to lecture slides or the book for more general information about how and why they are used. The rest of this section will only cover how to implement them in Weenix.

To implement shadow objects, it will be extremely helpful to understand how the methods of memory objects are called during a page lookup or dirty operation. If you don't remember this well from the last assignment, we recommend that you go back and either re-read the relevant sections of the last assignment or search through the code paths in question and draw a graph showing what functions in the page frame system call what functions in the related memory objects.

The main difference between shadow objects and other types of memory management objects is that shadow objects can be part of arbitrarily long chains of memory objects. Therefore, many calls to shadow objects will be rerouted to the object that is being shadowed, or occasionally to the root object in a tree of memory objects, which cannot be a shadow object. At a high level, this is similar to how file memory objects forward requests to the disk memory object, but in practice it ends up seeming a lot more recursive when implementing shadow objects since there is no translation layer as there was between file block numbers and disk block numbers. However, shadow objects are still responsible for storing some data and, more importantly, causing copy-on-write to work after a \texttt{fork()} has taken place.

One potential problem with shadow objects is that the chains must be cleaned up when the process that creates them exits to avoid temporary memory leaks. Ideally, this could happen at process exit. A process exiting might cause a shadow object's refcount to drop to one, at which point the pages attached to the object could be reassigned to the single shadow object beneath it, and the object itself could be deallocated. However, this would require the shadow object to know what its remaining child is and, at the moment, shadow objects do not maintain a list of their children.

This apparent design flaw leaves two other avenues for shadow chain cleanup. First, there is a shadow daemon known as \texttt{shadowd} which was built for this purpose. It should be invoked when the kernel is out of memory (this code is already written) or when a shadow object which can be cleaned up is created (you can tell this by checking for it when you remove either of its child shadow objects). To enable the shadow daemon, just set \texttt{SHADOWD=1} in the project environment settings. The second method would be to collapse shadow chains during \texttt{fork()}. This requires a relatively easy traversal of the forking process' object chains, where you shift the pages from any objects with a single child down to their child and then deallocate the objects.

\section{System Calls}

System calls are the only way user processes can communicate with the Weenix kernel directly. The way that system calls are generated from user space is by generating a software interrupt (using the x86 \texttt{int} instruction) with the arguments to the system call stored in the registers or on the stack. This causes an interrupt in the kernel, where the number in an agreed-upon register designates which system call is being used, and then the corresponding system call function is actually called to handle the request after the arguments have been parsed out of their registers. Most of the system calls have already been written for you, however, in order to give you some understanding of the process involved, you will need to write a few yourself.

\subsection{Kernel System Call Interface}

You will need to implement the kernel targets for \texttt{read()}, \texttt{write()}, and \texttt{getdents()}. While most of what you need to do should be pretty self-explanatory after reading through other system call implementations, you must also write two helper functions to check accesses to user memory from within the kernel.

\subsection{Accessing User Memory}

The code to handle traps and access user memory from the kernel has been written for you. However, many of these functions need to check to see if a region of user memory is a valid section of the process' address space. To check this, you will need to implement \texttt{addr\_perm()} and \texttt{range\_perm()}, which will rely on your virtual memory map code.

\subsection{Running Userland Programs}

Once you have implemented the page fault handler, anonymous objects, and \texttt{write()}, you should be able to run a variety of simple user-level programs. Of course, the first you should try to get running should also be one of the simplest, so we recommend \texttt{hello}, which should print ``Hello, world!'' to the screen. To run correctly, this will require a mostly-functional page fault handler to fill in pages as the process attempts to access them; otherwise, the operating system will probably go into an infinite loop, trying to access the same address over and over using the page fault handler, but never adding the correct entry to the page table. Some other simple programs that you should be able to run are \texttt{args} and \texttt{uname}.

If you are having trouble getting \texttt{hello} to run and suspect that your anonymous object or \texttt{write()} implementations might be at fault, you should try the \texttt{segfault} program instead and ensure that it exits with a status of \texttt{EFAULT}. If it doesn't (if, for instance, you run into the infinite loop problem described above), this means the bug is probably in your page fault handler.

After getting \texttt{hello} or \texttt{segfault} running, congratulations! You've just gotten your first userland program working! Celebration techniques are myriad, but we recommend dancing around a bit, and maybe taking a shower.

At this point, it will be useful to look at the \wlink{Debugging}{debugging} appendix covering how to inspect the progress of a user-level process using a debugger. Although you may not need it yet, we assume that you will want it very soon.

\subsection{VM-Related System Calls}

After you get some initial test programs running, you can start to think about implementing a variety of VM-related system calls. For the functions in this section, we recommend that you check out the documentation in the \texttt{man} pages for more information.

\texttt{mmap()} and \texttt{munmap()} are the most simple and obvious of the functions in this category. They allow user processes to map files into memory, create private or shared memory regions, and remove areas of their address space. The majority of these functions will end up being error-checking, since you wrote the main logic for them in the virtual memory map code.  Note that the Weenix memtests expect you to use the \texttt{VMMAP\_DIR\_HILO} flag.

\texttt{brk()} is similar in conceptual difficulty. Calling \texttt{brk()} changes the length of the memory region acting as the heap, but the pointer passed as an argument to \texttt{brk()} is not required to lie on a page boundary, and the beginning of the heap sometimes starts halfway through the last page of another memory region. This means that the edge cases for \texttt{brk()} can be a bit annoying, but there's nothing conceptually difficult to grasp here. There are some robust user-level tests for much of this functionality, so rather than spending a lot of time getting it right before testing, we recommend starting with something naive and gradually fixing it to pass the tests after you can run them in userland.

\section{\texttt{fork()}}

Although it is also a VM-related system call, \texttt{fork()} is an entirely different animal from \texttt{mmap()} and friends. A good implementation of the previous sections is essential; \texttt{fork()} is complicated enough without having to debug the rest of your VM code at the same time. The \texttt{man} pages, while useful as always, will not be as helpful for \texttt{fork()} as for the other system calls, so most of the documentation for \texttt{fork()} is given here.

\texttt{fork()} is a moderately complicated system call. We present it here as one long algorithm, but it will make your life much easier if you break it down into separate subroutines. Close attention to detail will help you; an under-debugged \texttt{fork()} can cause subtle instabilities and bugs later on.

Bugs in the virtual memory portion of \texttt{fork()} tend to cause bizarre behavior: user process memory may not be what it ought to be, so almost anything can happen. The user process may end up executing what should be data, jumping into the middle of a random subroutine, etc. These sorts of bugs are \emph{very} difficult to track down. For this reason, you should code more defensively than you may be used to. Assert everything you can, \texttt{panic()} at the first sign of trouble, and include apparently unnecessary sanity checks.

Above all, be sure you really understand the algorithm before you start coding. If you try to implement it before you understand what you are trying to do, you will write buggy code. In all likelihood you will then forget that you have written buggy code, and waste time debugging code that you should have thrown away. We know this because it has happened to us.

Note that these steps are not all in the correct order; consider the order in which you do them, keeping in mind what kind of cleanup you will need to do if one of them fails. Look out for steps which cannot be undone.

\begin{itemize}
    \item Create a new process using \texttt{proc\_create()}.
    \item Copy the \texttt{vmmap\_t} from the parent process into the child using \texttt{vmmap\_clone()} (which you should write if you haven't already). Remember to increase the reference counts on the underlying memory objects.
    \item For each private mapping in the original process, point the virtual memory areas of the new and old processes to two new shadow objects, which in turn should point to the original underlying memory object. This is how you know that pages corresponding to this mapping are copy-on-write. Be careful with reference counts. Also note that for shared mappings, there is no need to make a shadow object.
    \item Unmap the userland page table entries and flush the TLB using \texttt{pt\_unmap\_range()} and \texttt{tlb\_flush\_all()}. This is necessary because the parent process might still have some entries marked as ``writable'', but since we are implementing copy-on-write we would like access to these pages to cause a trap to our page fault handler so it can dirty the page, which will invoke the copy-on-write actions.
    \item Set up the new process thread context. You will need to set the following:
        \begin{itemize}
            \item \texttt{c\_pdptr} - the page table pointer
            \item \texttt{c\_eip} - function pointer for \texttt{userland\_entry()}
            \item \texttt{c\_esp} - the value returned by \texttt{fork\_setup\_stack()}
            \item \texttt{c\_kstack} - the top of the new thread's kernel stack
            \item \texttt{c\_kstacksz} - size of the new thread's kernel stack
        \end{itemize}
    \item Copy the file table of the parent into the child. Remember to use \texttt{fref()} here.
    \item Set the child's working directory to point to the parent's working directory. Once again, don't forget reference counts.
    \item Use \texttt{kthread\_clone()} (which you should write if you haven't yet) to copy the thread from the parent process into the child process.
    \item Set any other fields in the new process which need to be set.
    \item Make the new thread runnable, which will add it to the run queue.
\end{itemize}

Remember that the only difference between the parent and child processes is the return value of \texttt{fork()}. By 32-bit x86 convention, this value is returned in the \texttt{eax} register, which should be set in the context values of both threads. You should also revisit your implementation of the \texttt{proc\_exit()} function to make sure that your implementation is releasing all resources it should.

Note that a simpler, less correct implementation of \texttt{fork()} can function without actually using shadow objects, as long as you don't care what happens to whichever process (parent or child) wakes up last from the syscall. If you're having trouble getting shadow objects to work correctly, you can write \texttt{fork()} without them for testing purposes.

\section{Odds and Ends}

Finally, there are a number of other functions which you might remember seeing in earlier assignments spread throughout the kernel which you need to find and either write or update. These functions are all fairly small, but if you miss one, some things will break. Two examples are \texttt{special\_file\_mmap()} and \texttt{proc\_kill\_all()}. Once you get the last of these finished, you should be able to test your kernel with any binary file you find on the Weenix file system.

\section{Testing}

Testing your code at this point becomes rather difficult, since you must be able to create data and text in user land and execute it. This is an order of magnitude more difficult than creating kernel-mode threads as you have in past assignments. Thankfully most of the gory details have been taken care of for you (take a look at \texttt{kernel/api/elf32.c} and \texttt{user/ld-weenix/} if you are a masochist).

\subsection{Userland Tests}

Once you have functioning userland execution and a working \texttt{fork()} function, you are ready to complete your Weenix system by running the userland binaries we provide for you. All you need to do is call \texttt{kernel\_execve()} in your init process. You should execute the binary \texttt{/sbin/init}, which should start 3 shells (one in each terminal window). These shells will allow you to execute any of the provided binaries (roughly in order of difficulty):
\begin{itemize}
    \item \texttt{/usr/bin/segfault} - Even simpler than hello, this should just segfault on address \texttt{0x0}. Good if you're having a lot of trouble getting hello to run.
    \item \texttt{/usr/bin/hello} - A simple ``Hello world!'' test. Getting this to execute properly should be a big step in VM.
    \item \texttt{/usr/bin/args} - Prints command arguments.
    \item \texttt{/usr/bin/forktest} - Simple program which forks, and prints out everything of note.
    \item \texttt{/bin/uname} - Prints system information.
    \item \texttt{/bin/stat} - Prints information about a file.
    \item \texttt{/usr/bin/kshell} - Traps into the kernel and starts a kshell.
    \item \texttt{/bin/ls} - List the contents of a directory.
    \item \texttt{/sbin/halt} - Kills all processes and shuts the system down.
    \item \texttt{/usr/bin/wc} - Counts characters, words and lines.
    \item \texttt{/bin/hd} - Dumps input in hexadecimal.
    \item \texttt{/bin/sh} - The shell itself. Yay subshell fun!
    \item \texttt{/usr/bin/spin} - Executes ``\texttt{while(1);}''.
    \item \texttt{/usr/bin/forkbomb} - A forkbomb test which should theoretically run forever.
    \item \texttt{/usr/bin/stress} - A test to stress various parts of your system and then run a forkbomb.
    \item \texttt{check} - Contains checks for various test cases (this is a shell built-in command).
    \item \texttt{/usr/bin/vfstest} - Lots of VFS tests (error conditions, etc.).
    \item \texttt{/usr/bin/memtest} - Lots of memory management tests (mmap and brk).
    \item \texttt{/usr/bin/eatinodes} - Devours filesystem resources.
    \item \texttt{/usr/bin/eatmem} - Devours kernel memory.
    \item \texttt{/bin/ed} - \texttt{ed} is the standard text editor.
\end{itemize}

The shell also has a bunch of builtins. Type \texttt{help} in a shell to see a list of them. In particular, \texttt{repeat} and \texttt{parallel} can be very useful for stress testing your kernel.

\subsection{A Relatively Difficult Test Suite}

In addition to just having commands which work individually, you should be able to stress the hell out of your system. Run lots of difficult commands (\texttt{forkbomb}, \texttt{eatmem}, etc.) simultaneously, use different terminals at once, \texttt{halt} in the middle of all this, and so on. This type of testing can frequently be quite random, so here is a more systematic list of some things you can try. Make sure you start with a fresh disk.

\begin{itemize}
    \item \texttt{cat hamlet}
    \item \texttt{cat hamlet > hamlet2}
    \item \texttt{halt} (to shut down, then make sure the changes still exist on disk when you reboot)
    \item \texttt{rm hamlet}
    \item \texttt{cat /dev/null > foo}
    \item \texttt{ln foo bar}
    \item \texttt{cat README > foo}
    \item \texttt{cat bar}
    \item \texttt{check all} (do this three or four times in a row)
    \item \texttt{vfstest}
    \item \texttt{memtest} (do this three or four times in a row)
    \item \texttt{parallel vfstest -- vfstest}
    \item \texttt{parallel memtest -- memtest}
    \item \texttt{vfstest} and then \texttt{halt} while running (use \texttt{repeat} to re-run \texttt{vfstest} if it finishes too quickly)
    \item \texttt{memtest} and then \texttt{halt} while running
    \item \texttt{forkbomb} and then \texttt{halt} while running
    \item \texttt{forkbomb} and then \texttt{eatmem}
\end{itemize}

An easy way to make these tests harder is to check the kernel memory allocators for any leaks. See the \wlink{Debugging}{debugging} appendix to see how to do this. You may also want to test what happens when Weenix runs out of disk data blocks. To do this, set the \texttt{DISK\_BLOCKS} variable to 2048 if it is not already, and then re-run Weenix with a fresh disk and execute the commands below.

\begin{itemize}
    \item \texttt{cat hamlet >> hamlet} (this should reach the maximum file size and then exit)
    \item \texttt{vfstest} (this should work - the disk is not yet filled)
    \item \texttt{cat README >> README} (this should fill the disk but not quite reach the maximum file size)
    \item \texttt{vfstest} (this should fail to run most of the tests due to no disk data blocks being available)
\end{itemize}

\subsection{Dynamic Linking}

Once you feel everything is in good shape, enable the \texttt{DYNAMIC} variable and recompile the project from scratch. This will cause your userland libraries to be dynamically linked, which puts much more stress on your VM (especially \texttt{mmap()} code). This essentially adds another layer of indirection between the executable being run and the library calls it's attempting to run, where \texttt{ld-weenix} can link the library calls that are used into the original binary at runtime. Unfortunately, this also makes it even more difficult to debug what the user process is doing; if you are interested in setting breakpoints in the user process with dynamic linking, check out the \wlink{Debugging}{debugging} appendix for more information.

Turning dynamic linking on will make the above tests even more thorough. For instance, the test using \texttt{forkbomb} and \texttt{eatmem} simultaneously is notoriously hard to get right in the presence of dynamic linking - a phantom bug might cause your \texttt{pageoutd} to thrash back and forth between two pages if you're not careful.

\section{Conclusion}

We hope you've enjoyed working on Weenix and that you learned a lot. Good luck finishing your project, and don't forget to read the \wlink{Debugging}{debugging} appendix if you run into problems or need more ideas about where to look!
